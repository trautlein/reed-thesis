%Header
	\documentclass[12pt]{article}
	\usepackage{indentfirst}
	\usepackage{palatino}
	\usepackage{fullpage}
	\usepackage{microtype}
	\usepackage{ragged2e}
	\usepackage{setspace}
	
	\begin{document}
	
	
	
\noindent{\large
	Hans Trautlein 			\hfill Thesis Proposal\\
	Due: September 21, 2015  \hfill Thesis}
	
\center{
	\Large{The Effect of Nominations in Presidential Campaigns}} \\
\center{	\textbf{Advisor}			: 	Alex Montgomery \\
			\textbf{First Reader}	: 	Heather Hodges	
}

    
\bigskip \thispagestyle{empty}



\justify \onehalfspacing

I intend to write my thesis on the effects of political nominations in United States presidential campaigns. I am driven to this topic by the importance nominations seem to play in the entire process even though they seem to be ignored by political media and their importance is underplayed in the public arena. The issue was brought forward most by the important \emph{The Party Decides: Presidential Nominations Before and After Reform}.\footnote{Cohen, Marty et al. \emph{The Party Decides: Presidential Nominations Before and After Reform}. Chicago: University of Chicago Press, 2008.} This book will certainly be a guide to the ideas surrouding the nomination process and it's importance in American politics. I intend to use the dataset from the book\footnote{Marty Cohen - Data and Codebooks. "http://www.martycohen.net/5.html"} as well as some datasets incorporating new data, for example the following dataset from FiveThirtyEight.\footnote{data/endorsements/-june-30 at master - fivethirtyeight/data - GitHub. \\ "https://github.com/fivethirtyeight/data/tree/master/endorsements-june-30"}

I am expecting to use R throughout my thesis process as I intend for the thesis to be a primarily quantitative thesis. I also think I might examine GIS tools as examining geographic effects might be one way for me to find significance in this project. Ideas I am looking into, which could be superseded by the research of others, include the importance of party matching when endorsements are made, and how that could hold as a marker of elite polarization. I'm also curious in why the endorsements are made in the first place, so I might be curious in developing a model for what would encourage any politician to endorse another, perhaps looking at factors like time along with the many other variables that exist.

Finally, I want this thesis to primarily be a learning process about the tools involved and just a healthy introduction into how a researcher picks a topic and then dutifully inspects it.\footnote{274 words in body of proposal.}


\end{document}
